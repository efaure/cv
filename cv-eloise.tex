%!TEX TS-program = xelatex
\documentclass[]{friggeri-cv}
\usepackage{afterpage}
\usepackage{hyperref}
\usepackage{color}
\usepackage{xcolor}
\usepackage{smartdiagram}
\usepackage{fontspec}
% if you want to add fontawesome package
% you need to compile the tex file with LuaLaTeX
% References:
%   http://texdoc.net/texmf-dist/doc/latex/fontawesome/fontawesome.pdf
%   https://www.ctan.org/tex-archive/fonts/fontawesome?lang=en
%\usepackage{fontawesome}
\usepackage{metalogo}
\usepackage{dtklogos}
\usepackage[utf8]{inputenc}
\usepackage{tikz}
\usepackage{enumitem}
\usetikzlibrary{mindmap,shadows}
\hypersetup{
    pdftitle={},
    pdfauthor={},
    pdfsubject={},
    pdfkeywords={},
    colorlinks=false,           % no lik border color
    allbordercolors=white       % white border color for all
}
\smartdiagramset{
    bubble center node font = \footnotesize,
    bubble node font = \footnotesize,
    % specifies the minimum size of the bubble center node
    bubble center node size = 0.2cm,
    %  specifies the minimum size of the bubbles
    bubble node size = 0.5cm,
    % specifies which is the distance among the bubble center node and the other bubbles
    distance center/other bubbles = 0.3cm,
    % sets the distance from the text to the border of the bubble center node
    distance text center bubble = 0.5cm,
    % set center bubble color
    bubble center node color = pblue!60,
    % define the list of colors usable in the diagram
    set color list = {lightgray, materialcyan, orange, green, materialorange, materialteal, materialamber, materialindigo, materialgreen, materiallime},
    % sets the opacity at which the bubbles are shown
    bubble fill opacity = 0.6,
    % sets the opacity at which the bubble text is shown
    bubble text opacity = 1,
}

\RequirePackage{xcolor}
\definecolor{pblue}{HTML}{0395DE}

\begin{document}
	
\header{ Eloïse}{ Faure  \qquad }		
		 {\qquad Cloud \& DevOps Architect  }
      
% Fake text to add separator      
\fcolorbox{white}{gray}{\parbox{\dimexpr\textwidth-2\fboxsep-2\fboxrule}{%
.....
}}

% In the aside, each new line forces a line break
\begin{aside}
  \includegraphics[scale=0.36]{img/eloise.png}
  ~
  ~
  \section{Personnal information}
  ~    
 \ifdefined\PhoneNumber %
 {\includegraphics[width=0.4cm]{img/mobile_icon.pdf}} \textbf{ \PhoneNumber }  \vspace{6pt} 
  \fi %
% {\includegraphics[width=0.4cm]{img/mobile_icon.pdf}} \textbf{ +33 6 00 00 00 00 }  \vspace{6pt} 
\raisebox{-2\lineskip}{\includegraphics[width=0.4cm]{img/mail_icon.pdf}}  \href{mailto:faure.eloise@gmail.com}{\textbf{faure.eloise@}gmail.com} \vspace{-4pt}
%
\raisebox{-2\lineskip}{\includegraphics[width=0.4cm]{img/linkedin.png}}  \href{https://fr.linkedin.com/in/eloisefaure}{ /eloisefaure} \vspace{7pt}
%
{\includegraphics[width=0.4cm]{img/localisation.png}}  Argenteuil, France \vspace{6pt}
%
{\includegraphics[width=0.4cm]{img/nationality.png}}    \enspace French 
~
~
    \section{Experience}
    ~
    	\textbf{13 years} experience in 	\textbf{development} and 	\textbf{DevOps}
    	~
    	~
  % use  \hspace{} or \vspace{} to change bubble size, if needed
    \section{Main skills}
    ~
    ~
    ~
  \smartdiagram[bubble diagram]{
  	\scalebox{1.3}{\textbf{Devops}},
  	\scalebox{1.3}{  \textbf{Cloud \hspace{1mm}}},
  	\textbf{Container},
  	\textbf{Automation},
  	\textbf{Security\hspace{2mm}},
  	\textbf{Monitoring},
  	\scalebox{1.3}{ \hspace{1mm} \textbf{Dev \hspace{1mm}}}
  }
    ~
    ~
    \section{Languages}
    ~
    \textbf{French}\includegraphics[scale=0.40]{img/5stars.png}
    ~
    \textbf{English}\includegraphics[scale=0.40]{img/4stars.png}
    ~    
\end{aside}
~
\newlength{\parsepsave}
\section{\ About\ me} 
 \qquad I am a dev who turned devops. 
 
 \vspace{-4pt}
 \qquad  As a consultant and then DevOps architect at Red Hat I began my path to the Ops side. Naturally I completed my range of existing skills in dev and CI by pushing my way through new subjects like infrastructure, cloud, automation, orchestration, security, high availability, monitoring, performance...
 
 \vspace{-4pt}
 \qquad   I appreciate being able to get a global understanding of a project, finding the best way to achieve final goal taking into account specific constraint and caring of choices consequences. 
 
 \vspace{-4pt}
 \qquad  I'm an Open Source believer.  
\section{Experience}
\setlength{\parsepsave}{\parsep}% Store \parsep
\setlength{\parsep}{8.0pt}% New/update \parsep
\begin{entrylist}	
	\entryDate
%		{08/2018 }
		{ Aug 2018 }
		{Cloud \& DevOps Architect}
		{Freelance}
		{	\vspace{-10pt}
			\begin{itemize}[wide]
			\item Support customer defining and implementing \textbf{OpenShift} production architecture on \textbf{AWS} and on \textbf{hybrid cloud (on premise and Azure)}.  
			\item Implementation of automation (\textbf{Terraform, Ansible}), CI (Jenkins) and ramping up of teams skills.  \
		 \item Customers : Sopra Steria, Storengy
		\end{itemize}
		}
	\entryDate
	%	{12/2017 \\ 08/2018 }
		{Dec 2017 - \\ Aug 2018 }
		{Cloud \& DevOps Architect}
		{Red Hat}
		{	\vspace{-10pt}
			\begin{itemize}[wide]
			\item Definition of Openshift architecture on Cloud or on premise. 
			\item Advise and assist customers on key steps on their move to DevOps, HA, resiliency, \textbf{Infrastructure as Code}, \textbf{monitoring}, automation, CI, \textbf{security}... 
			\item Some customers : AXA, La poste, Amadeus, Natixis, Airbus, SNCF, MyCOM Groupama...    
		\end{itemize}
		}
	\entryDate
%		{10/2015 \\ 12/2017}
		{Oct 2015 - \\ Dec 2017}
		{Midleware \& DevOps Consultant}
		{Red Hat}
		{	\vspace{-10pt}
			\begin{itemize}[wide]
			\item \textbf{OpenShift} deployment adapted to customer constraints and automation. 
			\item Assist customers on subject around Red Hat technology, DevOps, application development, JBoss, \textbf{Docker, Kubernetes}, monitoring, security... 
			\item DevOps and \textbf{CI/CD} coaching with practical implementation. 
			\item Training and knowledge transfer.
		\end{itemize}
		}	
	\entryDate
	% 	{06/2013 \\  09/2015}
	 	{Jun 2013 - \\  Sept 2015}
		{Senior Software Engineer}
		{STET}
		{	\vspace{-10pt}
			\begin{itemize}[wide]
			\item Migration from IBM to \textbf{Open Source} (Application server, SSO stack)			
			\item \textbf{Performance} benchmark 
		 	\item Automation tools development to deploy middleware 
		 	\item CI tools development 
	 	\end{itemize}
	 	} 
	\entryDate
	%	{09/2011 \\ 04/2013}
		{Sept 2011 - \\ Apr 2013}
		{Technical Project Manager}
		{Saint Gobain}
		{
			Leading multiple project from development and testing to business delivery. Working with offshore indian team. Managing all aspect of project including budget, application architecture and troubleshooting }
	\entryDate
%		{10/2009 \\ 07/2011}    
		{Oct 2009 - \\ Jul 2011}    
		{Technical Project Leader}
		{Euler Hermes}
		{Client website maintenance in a team of 5 developers }
	\entryDate
%		{11/2007 \\ 10/2009}
		{Nov 2007 - \\ Oct 2009}
		{Software Engineer}
		{AXA Banque}
		{Working in the IT Architect team on the rebuild of AXA Banque client website}
	\end{entrylist}
\setlength{\parsep}{\parsepsave}% Restore \parsep
\\
\newpage

\begin{aside}
~
~
~
\section{DevOps skills}
~  
\smartdiagram[bubble diagram]{
	\scalebox{1.2}{\textbf{Container}},
	\scalebox{1.2}{\textbf{\hspace{1mm}Docker\hspace{1mm}}},
	\textbf{Jenkins},
	\textbf{Infra}\\\textbf{as Code},
	\textbf{Kubernetes},
	\textbf{OpenShift},
	\scalebox{1.2}{\textbf{\hspace{1mm}CI/CD\hspace{1mm}}},
	\textbf{Linux}
}
~  
\section{Programming skills}
~  
\smartdiagram[bubble diagram]{
	\scalebox{1.3}{\textbf{Java}},
	\textbf{Jenkins}\\\textbf{pipeline},
	\textbf{Scala},
	\textbf{Ansible},
	\textbf{Terraform},
	\scalebox{1.2}{\textbf{\hspace{1mm}Bash\hspace{1mm}}},
	PHP\\JS
}
    ~
   	\section{Interest}
   	~  
    	\textbf{Yoga}
    	\textbf{Travel} 
    	\textbf{Reading} 
    	\textbf{Zero waste}        	
    	~
\end{aside}
\section{Education}
\begin{entrylist}
	\entry
	{2007}
	{ESIEA Paris - Ecole Supérieure d’Informatique \\  Electronique Automatique}
	{France}
	{Master's degree in Engineering}
	\entry
	{2005}
	{DCU - Dublin City University}
	{Ireland}
	{Computer Science – One semester exchange}	
\end{entrylist}
\section{Certifications}        
    \textbf{\large{Red Hat (id: 160-101-879 )} \vspace{0.2cm}\\}        
	\begin{entrylist}
	
	\setlength{\parsepsave}{\parsep}% Store \parsep
	\setlength{\parsep}{-3pt}% New/update \parsep
	
	\entryNoDetail
		{Nov 2016}
		{Red Hat Certified Specialist in Ansible Automation}
	    {EX407}
	    \entryNoDetail
	    {Jul 2016}
	    {Red Hat Certified Specialist in OpenShift Administration}
	    {EX280}
	    \entryNoDetail
	    {Jul 2016}
	    {RHCSA : Red Hat Certified System Administrator }
	    {EX200}   
	    \entryNoDetail
	    {Jun 2016}
	    {Red Hat Certificate of Expertise in Containerized Application \\ Development}
	    {EX276}
	\end{entrylist}    
	\textbf{\large{Coursera} \vspace{0.2cm}\\}    
	\begin{entrylist}
	\entryNoDetail
		{May 2015} 
		{Principles of Reactive Programming }
		{EPFL}
		\entryNoDetail
		{Nov 2014}
		{Functional Programming Principles in Scala}
		{EPFL}
	\end{entrylist}    
\section{Speaker}  
\setlength{\parsepsave}{\parsep}% Store \parsep
\setlength{\parsep}{3pt}% New/update \parsep
	\begin{entrylist}	
		\entrySpeaker
		{May 2018}
		{OpenShift Meetup France}
		{Paris}
		{ {\includegraphics[width=0.3cm]{img/link.png}} \href{https://www.meetup.com/fr-FR/OpenShift-France/events/250776251/} { Comment Openshift vous facilite Kubernetes }}
		\entrySpeaker
		{May 2018}
		{RivieraDev}
		{Nice}
		{ {\includegraphics[width=0.3cm]{img/link.png}} \href{https://2018.rivieradev.fr/session/370}
		{Comment Openshift vous facilite Kubernetes }}		
		\entrySpeaker
		{Jan 2018 }
		{Duchess France Meetup}
		{Paris}
	 	{ {\includegraphics[width=0.3cm]{img/link.png}}
		 	\href{https://www.duchess-france.org/soiree-docker-kubernetes-openshift-eloise-faure/} { Docker Kubernetes and OpenShift }}
		\entrySpeaker
		{Sept 2017}
		{Red Hat Tech Exchange}
		{Barcelona}
		{OpenShift in Production: 7 traps not to fall in}	
		\entrySpeaker
		{Jun 2017}
		{Paris Container Day}
		{Paris}
		 {  {\includegraphics[width=0.3cm]{img/youtube.png}} \href{https://www.youtube.com/watch?v=zK2oqZhhPSI} { OpenShift en production: 7 pièges à éviter } }		
	\end{entrylist}

\begin{bottom}
\qquad \qquad \qquad \qquad %
\includegraphics[height=12mm]{logo/terraform-logo.png} \qquad \qquad \qquad  %
\includegraphics[height=12mm]{logo/ansible-logo.png} \qquad \qquad \qquad %
\includegraphics[height=12mm]{logo/prometheus-logo.png} \qquad \qquad  %
\includegraphics[height=12mm]{logo/grafana-logo.png} \qquad \qquad  %
~
~
\qquad \qquad \qquad \qquad \qquad \qquad \qquad %
\includegraphics[height=13mm]{logo/kubernetes-logo.png}  \qquad \qquad  %
\includegraphics[height=13mm]{logo/openshift-logo.png}    \qquad \qquad  %
\includegraphics[height=13mm]{logo/docker-logo.png}  \qquad \qquad  %
~
~
\qquad \qquad \qquad \qquad %
\includegraphics[height=13mm]{logo/java-logo.png} \qquad  \qquad  %
\includegraphics[height=12mm]{logo/git-logo.png} \qquad\qquad  \qquad %
\includegraphics[height=12mm]{logo/scala-logo.png} \qquad \qquad %
\includegraphics[height=12mm]{logo/bash-logo.png} \qquad \qquad %
\includegraphics[height=12mm]{logo/jenkins-logo.png} \qquad  \qquad  %
~
~
\qquad \qquad%
\includegraphics[height=12mm]{logo/linux-logo.png} \qquad \qquad \qquad  %
\includegraphics[height=12mm]{logo/fedora-logo.png} \qquad \qquad \qquad %
\includegraphics[height=12mm]{logo/redhat-logo.png} \qquad \qquad  \qquad%
\includegraphics[height=12mm]{logo/aws-logo.png}\qquad\qquad  \qquad %
\includegraphics[height=12mm]{logo/azure-logo.png} 
~ %
\begin{flushleft} %
	\emph{Eloise Faure} %
\end{flushleft} %
\end{bottom}
\end{document}